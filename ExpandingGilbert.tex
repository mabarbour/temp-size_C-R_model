\documentclass[letter,12pt]{article}
\usepackage{fontenc}
\usepackage[english]{babel}

\usepackage{fullpage}
\usepackage{amsmath,amssymb}
\usepackage[colorlinks=true,linkcolor=blue,citecolor=blue]{hyperref}
\bibliographystyle{plain}

\title{The temperature-size rule may stabilise consumer-resource dynamics under warming}
\date{\today}
\author{UBC MST group}

\begin{document}
\maketitle
\tableofcontents

%%%%%%%%%%%%%%%%%%%%%%%%%%%%%%%%%%%%%%%%%%%%%%%%%%%%%%%
\section*{Abstract}

%%%%%%%%%%%%%%%%%%%%%%%%%%%%%%%%%%%%%%%%%%%%%%%%%%%%%%%
\textbf{Keywords:} Metabolic theory, predator-prey, plant-herbivore, body size

%%%%%%%%%%%%%%%%%%%%%%%%%%%%%%%%%%%%%%%%%%%%%%%%%%%%%%%
\section{Introduction}

Temperature and body size determine many biological rates \cite{West1997,Gillooly2001}.
How these factors individually influence consumer-resource dynamics has already been demonstrated \cite{Gilbert2014,DeLong2015}.
How they act in symphony has not yet been investigated.

Temperature also affects body size through the temperature-size rule \cite{Atkinson1994}.
Thus, temperature directly and indirectly affects ecosystem dynamics.
These two pathways have the potential to reinforce or counteract one another, with population level consequences.

Here we model a simple consumer-resource interaction, with population dynamic parameters that depend on temperature and body size, and body sizes that depend on temperature.
We ask how including dependencies on both temperature and body size, as well as incorporating the temperature-size rule, affects the response of the consumer-resource dynamics to increasing temperature.

%%%%%%%%%%%%%%%%%%%%%%%%%%%%%%%%%%%%%%%%%%%%%%%%%%%%%%%
\section{Methods and results}

\subsection{The underlying consumer-resource dynamics}

We begin, like \cite{Gilbert2014}, with the Rosenzweig-MacArthur equations \cite{Rosenzweig1963}
\begin{equation}\label{eq:RM}
\begin{aligned}
\frac{\mathrm{d}R}{\mathrm{d}t} =& r R \left(1 - \frac{R}{K} \right) - f(R) C\\
\frac{\mathrm{d}C}{\mathrm{d}t} =& e f(R) C - m C,
\end{aligned}
\end{equation}
which describe the rates of change in total resource $R\in[0,K]$ and consumer $C\geq0$ biomass with time $t$.

In the absence of consumers, $C=0$, the resource grows logistically, with intrinsic growth rate $r\geq0$ and carrying capacity $K>0$.
The intrinsic growth rate describes the rate at which resource biomass increases without consumers when the resource is rare, $R\approx0$.
The carrying capacity is the equilibrium biomass of the resource without consumers.

Resource biomass is consumed by consumers at a rate $f(R)$, where $f(R)\geq0$ is called the functional response.
Of the biomass consumed, the unitless conversion efficiency parameter $e\in[0,1]$ determines the proportion of resource biomass that is directly converted into consumer biomass.
Consumers die at constant per capita mortality rate $m\geq0$.

An equilibrium is reached when the two rates of change in Equation \eqref{eq:RM} are zero, and solving the system at this point gives equilibrium resource $\hat{R}$ and consumer $\hat{C}$ biomass.
There are three equilibria for this system: total extinction $(R,C) = (0,0)$, consumer extinction $(R,C)=(K,0)$, and coexistence $(R,C)=(\hat{R},\hat{C})$, the latter with $\hat{R}>0$ and $\hat{C}>0$.
We are primarily concerned with the latter equilibrium, as that is presumably the equilibrium current consumer-resource systems are near.
At this coexistence equilibrium one can calculate the ratio of consumer to resource biomass, $\hat{C}/\hat{R}$, and also perform a linear stability analysis to derive the leading eigenvalue $\lambda$, which determines if (and how readily) the system, when perturbed a small amount from this equilibrium, will return to it (see the supplementary \texttt{Mathematica} file for details).
Together these two measures tell us how biomass is partitioned and how stable this partitioning is.

As explained in \cite{Gilbert2014}, two aggregates well describe the dynamics of this system.
The first is $m R/(e f(R))$, which describes consumer growth and is the consumer zero-net growth isocline.
The second aggregate is $K$, the equilibrium resource biomass in the absence of consumers.
Dividing the second aggregate by the first gives a measure that defines the biomass potential of the resource that is converted into consumer biomass, $B_{CR} = e f(R) K / (R m)$.
Note that we have modified the presentation of \cite{Gilbert2014} by subsuming $R$ into $f(R)$ such that $f(R)$ is  a rate and thus more easily interpreted.
This does not alter the results.

In what follows we will examine how our three measures, $B_{CR}$, $\hat{C}/\hat{R}$, and $\lambda$, change with temperature.

\subsection{Temperature dependence}



\subsection{Mass dependence}

\subsection{The temperature-size rule}

%%%%%%%%%%%%%%%%%%%%%%%%%%%%%%%%%%%%%%%%%%%%%%%%%%%%%%%
\section{Discussion}

%%%%%%%%%%%%%%%%%%%%%%%%%%%%%%%%%%%%%%%%%%%%%%%%%%%%%%%
\bibliography{BIB/library}

\end{document}
